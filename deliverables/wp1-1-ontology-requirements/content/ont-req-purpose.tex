\section{Introduction}\label{ariaid-title13}

This document aims to analyse and formulate the requirements of the
EUR-Lex Legal Analysis team, focusing mainly on the data model and potential future applications. It also provides an approach for transposing the Legal Analysis Methodology (LAM) from a plain text document into structured data with semi-formal and formal underpinning.

The benefit of having the LAM represented in a structured form is that it enables automation of multiple processes, such as document classification, metadata validation and metadata enrichment, which currently are performed manually by the OP staff or by external contractors. Such an automation can lead to significant reductions of cost and reduce the time needed for performing these processes.

This document does not intend to provide a detailed functional specification for every envisaged element but rather provide a general direction and describe the path for reaching different business and technical objectives. The main concern, at this stage is to describe how a LAM ontology can be created starting from the current state of affairs presented in the next section.
