
\section{Intended uses}\label{ariaid-title1}

There are several use cases motivating the current modelling exercise.
This section briefly briefly describes them.

\subsection{Maintenance of the LAM
description}\label{maintenance-of-the-lam-description}

Currently LAM is maintained as a set of Word and Excel documents, this
makes the editing cumbersome and most importantly error prone. Making
references to formal properties in CDM ontology, manually tracking rules
and dependencies between LAM elements becomes increasingly difficult as
it becomes larger. Moreover, sharing the documents, collecting input and
the interaction with stakeholders and partners becomes difficult.

There is a need for providing an interactive documentation of Legal
Analysis Methodology, which should include a complete description of
document classes, properties, metadata and constraints; and enable an
easy navigation based on the dependencies between them. This interactive
documentation should also enable collecting feedback, corrections and
suggestions for improvement on any part of the methodology. It should
serve as the main point of access for the LAM for consultation purposes
for both experts and lay people.

\subsection{Modelling and structuring LAM
description}\label{modelling-and-structuring-lam-description}

Currently LAM is described in informal manner, therefore it is not
possible to automate or implement any automated processes relying on it.
To enable automatisation of any sort, the domain model must be created
first and formalised in a machine readable format. The model provides a
vocabulary for describing LAM concepts and the structural connections
between them.

The modelling need is situated at two levels of abstraction or perhaps
even two meta-levels. First, the LAM documentation covers descriptions
of classes of legal documents, so there is a need to formalise these
descriptions, lets call it Legal Document definition model. Second,
there is a need to formalise how the descriptions of classes should be
structured; let's call it the LAM meta-model.

\subsection{Consistency checking of the LAM
description}\label{consistency-checking-of-the-lam-description}

It is not easy to verify whether the document class is consistent with
other classes and properties described in the model. The model should
enable performing tests to determine if the data has any internal
conflicts. The exact type of conflicts is not determined at this point
and they may refer to redundancy, cycles, or contradiction detection.

\subsection{Automatic validation and curation of the document
descriptions}\label{automatic-validation-and-curation-of-the-document-descriptions}

The legal document descriptions instantiating LAM model should be
conform to model instantiation rules and fit specific data shapes. The
model should enable automatic verification and validation of instance
data.

In a similar manner, it should be possible to derive, for invalid
documents, what parts need to be modified in order to correct the
description.
