\section{Application profile}\label{ariaid-title11}

This section presents the application profile describing each class and data type. 
The class descriptions also comprise a list of relevant properties and the configuration in which they may appear in the data.

\subsection{lam:DocumentProperty}

This class aims at describing properties used for describing legal
documents. It aims at capturing the rules and constraints on permitted
annotations and controlled lists of values.

This is a proxy class for the formal properties. It is aimed at
capturing the documental and partially the constraints imposed on this
property.

\textbf{Superclasses:}

\begin{itemize}
\itemsep1pt\parskip0pt\parsep0pt
\item
  \emph{skos:Concept}
\item
  \emph{sh:PropertyShape}
\end{itemize}

{
	%	\fontsize{10pt}{10pt}
	\footnotesize
	\selectfont%\tabcolsep=3pt % hold it local
	\begin{longtable}[c]{@{}p{3cm}p{2cm}p{2cm}p{7.8cm}@{}}
		\toprule\addlinespace
		Name & Type & Cardinality & Definition
		\\\addlinespace
		\midrule\endhead
		sh:class & rdfs:Resource & 0..1 & The condition specified by sh:class is
		that each value node is a SHACL instance of a given type. This property
		constraints the range of the values that the property may take.
		\\\addlinespace
		sh:path & rdfs:Resource & 1..1 & The property that is formally used in
		describing the legal documents. Usually a CDM property.
		\\\addlinespace
		\bottomrule
		\addlinespace
		\caption{Properties of the lam:DocumentProperty class}
	\end{longtable}
}

\subsection{lam:LegalDocumentClass}

This class aims at describing a set of legal documents that have in
common a set of properties. It aims at capturing the rules and
constraints on legal document properties, metadata elements, and
cardinality.

\textbf{Superclasses:}

\begin{itemize}
\itemsep1pt\parskip0pt\parsep0pt
\item
  \emph{skos:Concept}
\end{itemize}

{
	%	\fontsize{10pt}{10pt}
	\footnotesize
	\selectfont%\tabcolsep=3pt % hold it local
	\begin{longtable}[c]{@{}p{3cm}p{3cm}p{2cm}p{5.8cm}@{}}
		\toprule\addlinespace
		Name & Type & Cardinality & Definition
		\\\addlinespace
		\midrule\endhead
		lam:hasProperty​Configuration​ & lam:Property​Configuration​ & 1..*​ &
		The constraint definition for a legal document property. ​
		\\\addlinespace
		lam:classify​With​ & lam:Classiffication​Hint​ & 0..*​ &
		\\\addlinespace
		\bottomrule
		\addlinespace
		\caption{Properties of the lam:LegalDocumentClass class}
	\end{longtable}
}

\subsection{lam:LegalDocumentGroup}

A way of grouping legal document classes.

\textbf{Superclasses:}

\begin{itemize}
\itemsep1pt\parskip0pt\parsep0pt
\item
  \emph{skos:Collection}
\end{itemize}

\subsection{lam:LegalPropertyGroup}

A way of grouping properties of the legal documents.

\textbf{Superclasses:}

\begin{itemize}
\itemsep1pt\parskip0pt\parsep0pt
\item
  \emph{skos:Collection}
\end{itemize}

\subsection{lam:AnnotationConfiguration}

Annotation configurations define the constraints on the annotations of
the legal documents.

\textbf{Superclasses:}

\begin{itemize}
\itemsep1pt\parskip0pt\parsep0pt
\item
  \emph{sh:PropertyShape}
\end{itemize}

{
	%	\fontsize{10pt}{10pt}
	\footnotesize
	\selectfont%\tabcolsep=3pt % hold it local
	\begin{longtable}[c]{@{}p{3cm}p{2cm}p{2cm}p{7.8cm}@{}}
		\toprule\addlinespace
		Name & Type & Cardinality & Definition
		\\\addlinespace
		\midrule\endhead
		sh:class & rdfs:Resource & 1..1 & The condition specified by sh:class is
		that each value node is a SHACL instance of a given type. This property
		constraints the range of the values that the property may take.
		\\\addlinespace
		sh:hasValue & rdfs:Resource & 1..1 & sh:hasValue specifies the condition
		that at least one value node is equal to the given RDF term.
		\\\addlinespace
		\bottomrule
		\addlinespace
		\caption{Properties of the lam:AnnotationConfiguration class}
	\end{longtable}
}

\subsection{lam:ClassifficationHint}

This class defines how a document may be automatically assigned as a
member of a given class of legal documents.

\textbf{Superclasses:}

\begin{itemize}
\itemsep1pt\parskip0pt\parsep0pt
\item
  \emph{sh:PropertyShape}
\end{itemize}

{
	%	\fontsize{10pt}{10pt}
	\footnotesize
	\selectfont%\tabcolsep=3pt % hold it local
	\begin{longtable}[c]{@{}p{3cm}p{2cm}p{2cm}p{7.8cm}@{}}
		\toprule\addlinespace
		Name & Type & Cardinality & Definition
		\\\addlinespace
		\midrule\endhead
		sh:flags & rdfs:Literal & 0..1 & An optional string of flags accepting
		SPARQL RegEx flags.
		\\\addlinespace
		sh:pattern & rdfs:Literal & 1..* & sh:pattern specifies a regular
		expression that each value node matches to satisfy the condition.
		
		It can be plain text or RegEx pattern matching the value of a given
		lam:property.
		\\\addlinespace
		\bottomrule
		\addlinespace
		\caption{Properties of the lam:ClassifficationHint class}
	\end{longtable}
}

\subsection{lam:Propertyconfiguration}

The Property Configuration define constraints on a lam:DocumentProperty
for a given lam:DocumentClass

\textbf{Superclasses:}

\begin{itemize}
\itemsep1pt\parskip0pt\parsep0pt
\item
  \emph{sh:PropertyShape}
\end{itemize}

{
	%	\fontsize{10pt}{10pt}
	\footnotesize
	\selectfont%\tabcolsep=3pt % hold it local
	\begin{longtable}[c]{@{}p{3cm}p{2cm}p{2cm}p{7.8cm}@{}}
		\toprule\addlinespace
		Name & Type & Cardinality & Definition
		\\\addlinespace
		\midrule\endhead
		sh:hasValue & rdfs:Resource & 0..1 & sh:hasValue specifies the condition
		that at least one value node is equal to the given RDF term.
		\\\addlinespace
		sh:maxCount & xsd:int & 0..1 & Maximum cardinality constraint. Formally
		sh:maxCount specifies the maximum number of value nodes that satisfy the
		condition.
		\\\addlinespace
		sh:minCount & xsd:int & 0..1 & Minimum cardinality constraint. Formally
		sh:minCount specifies the minimum number of value nodes that satisfy the
		condition. If the minimum cardinality value is 0 then this constraint is
		always satisfied and so may be omitted.
		\\\addlinespace
		skos:editorialNote & rdfs:Literal & 0..* & A note for an editor,
		translator or maintainer of the vocabulary.
		\\\addlinespace
		\bottomrule
		\addlinespace
		\caption{Properties of the lam:Propertyconfiguration class}
	\end{longtable}
}

\subsection{skos:ConceptScheme}

A SKOS concept scheme can be viewed as an aggregation of one or more
SKOS concepts. Semantic relationships (links) between those concepts may
also be viewed as part of a concept scheme. This definition is, however,
meant to be suggestive rather than restrictive, and there is some
flexibility in the formal data model stated below.

Thesauri, classification schemes, subject heading lists, taxonomies,
`folksonomies', and other types of controlled vocabulary are all
examples of concept schemes. Concept schemes are also embedded in
glossaries and terminologies.

{
	%	\fontsize{10pt}{10pt}
	\footnotesize
	\selectfont%\tabcolsep=3pt % hold it local
	\begin{longtable}[c]{@{}p{3cm}p{2cm}p{2cm}p{7.8cm}@{}}
		\toprule\addlinespace
		Name & Type & Cardinality & Definition
		\\\addlinespace
		\midrule\endhead
		owl:versionInfo & rdfs:Literal & 0..1 & An owl:versionInfo statement
		generally has as its object a string giving information about this
		version. This statement does not contribute to the logical meaning of
		the resource.
		\\\addlinespace
		skos:prefLabel & rdfs:Literal & 1..1 & The preferred lexical label for a
		resource, in a given language. No two concepts in the same concept
		scheme may have the same preferred label in a given language.
		\\\addlinespace
		\bottomrule
		\addlinespace
		\caption{Properties of the skos:ConceptScheme class}
	\end{longtable}
}

\subsection{skos:Concept}

A SKOS concept can be viewed as an idea or notion; a unit of thought.
However, what constitutes a unit of thought is subjective, and this
definition is meant to be suggestive, rather than restrictive.

The notion of a SKOS concept is useful when describing the conceptual or
intellectual structure of a knowledge organization system, and when
referring to specific ideas or meanings established within a KOS.

Note that, because SKOS is designed to be a vehicle for representing
semi-formal KOS, such as thesauri and classification schemes, a certain
amount of flexibility has been built in to the formal definition of this
class.

{
	%	\fontsize{10pt}{10pt}
	\footnotesize
	\selectfont%\tabcolsep=3pt % hold it local
	\begin{longtable}[c]{@{}p{3cm}p{2cm}p{2cm}p{7.8cm}@{}}
		\toprule\addlinespace
		Name & Type & Cardinality & Definition
		\\\addlinespace
		\midrule\endhead
		owl:deprecated & xsd:boolean & 0..1 & States whether the resource is
		current or deprecated. By deprecating a resource, it means that it
		should not be used in new documents.
		
		Deprecation is a feature commonly used in versioning software to
		indicate that a particular feature is preserved for
		backward-compatibility purposes, but may be phased out in the future.
		\\\addlinespace
		skos:altLabel & rdfs:Literal & 0..* & An alternative lexical label for a
		resource. Acronyms, abbreviations, spelling variants, and irregular
		plural/singular forms may be included among the alternative labels for a
		concept.
		\\\addlinespace
		skos:changeNote & rdfs:Literal & 0..* & A note about a modification to a
		concept.
		\\\addlinespace
		skos:definition & rdfs:Literal & 0..1 & A statement or formal
		explanation of the meaning of a concept.
		\\\addlinespace
		skos:editorialNote & rdfs:Literal & 0..* & A note for an editor,
		translator or maintainer of the vocabulary.
		\\\addlinespace
		skos:example & rdfs:Literal & 0..* & An example of the use of a concept.
		\\\addlinespace
		skos:notation & rdfs:Literal & 0..* & A notation is a string of
		characters such as ``T58.5'' or ``303.4833'' used to uniquely identify a
		concept within the scope of a given concept scheme or within a specified
		context.
		\\\addlinespace
		skos:prefLabel & rdfs:Literal & 1..1 & The preferred lexical label for a
		resource, in a given language. No two concepts in the same concept
		scheme may have the same preferred label in a given language.
		\\\addlinespace
		skos:scopeNote & rdfs:Literal & 0..1 & A note that helps to clarify the
		meaning of a concept.
		\\\addlinespace
		skos:broader & skos:Concept & 0..* & A concept that is more general in
		meaning. Broader concepts are typically rendered as parents in a concept
		hierarchy (tree).
		\\\addlinespace
		dct:isReplacedBy & skos:Concept & 0..* & A related resource that
		supplants, displaces, or supersedes the described resource.
		\\\addlinespace
		skos:related & skos:Concept & 0..* & A concept with which there is an
		associative semantic relationship.
		\\\addlinespace
		\bottomrule
		\addlinespace
		\caption{Properties of the skos:Concept class}
	\end{longtable}
}

\subsection{skos:Collection}

A meaningful collection of concepts. Labeled collections can be used
with collectible semantic relation properties i.e. skos:narrower, where
you would like a set of concepts to be displayed under a ``node label'' in
the hierarchy.

{
	%	\fontsize{10pt}{10pt}
	\footnotesize
	\selectfont%\tabcolsep=3pt % hold it local
	\begin{longtable}[c]{@{}p{3cm}p{2cm}p{2cm}p{7.8cm}@{}}
		\toprule\addlinespace
		Name & Type & Cardinality & Definition
		\\\addlinespace
		\midrule\endhead
		skos:definition & rdfs:Literal & 0..1 & A statement or formal
		explanation of the meaning of a collection.
		\\\addlinespace
		skos:prefLabel & rdfs:Literal & 1..1 & The preferred lexical label for a
		resource, in a given language. No two concepts in the same concept
		scheme may have the same preferred label in a given language.
		\\\addlinespace
		skos:member & skos:Collection & 0..* & A member of a collection.
		\\\addlinespace
		\bottomrule
		\addlinespace
		\caption{Properties of the skos:Collection class}
	\end{longtable}
}

\subsection{rdfs:Resource}

All things described by RDF are called resources, and are instances of
the class rdfs:Resource. This is the class of everything.

\subsection{sh:PropertyShape}

A property shape is a shape in the shapes graph that is the subject of a
triple that has sh:path as its predicate. A shape has at most one value
for sh:path. Each value of sh:path in a shape must be a well-formed
SHACL property path. 

In LAM-SKOS-AP the role of sh:path is taken by lam:path.

{
	%	\fontsize{10pt}{10pt}
	\footnotesize
	\selectfont%\tabcolsep=3pt % hold it local
	\begin{longtable}[c]{@{}p{3cm}p{2cm}p{2cm}p{7.8cm}@{}}
		\toprule\addlinespace
		Name & Type & Cardinality & Definition
		\\\addlinespace
		\midrule\endhead
		sh:description & rdfs:Literal & 0..* & Property shape may have values
		for sh:description to provide descriptions of the property in the given
		context.
		\\\addlinespace
		sh:name & rdfs:Literal & 1..1 & Property shapes may have one or more
		values for sh:name to provide human-readable labels for the property in
		the target where it appears.
		\\\addlinespace
		\bottomrule
		\addlinespace
		\caption{Properties of the sh:PropertyShape class}
	\end{longtable}
}

\subsection{euvoc:XlNotation}

A notation is a string of characters used to uniquely identify a concept
within a specified context.

Like the skosxl:Label class reifies SKOS label statements, XlNotation
reifies SKOS notation statements. This class permits, if needed, to
maintain the historical view of the values and add additional provenance
descriptions.
{
	%	\fontsize{10pt}{10pt}
	\footnotesize
	\selectfont%\tabcolsep=3pt % hold it local
	\begin{longtable}[c]{@{}p{3cm}p{2cm}p{2cm}p{7.8cm}@{}}
		\toprule\addlinespace
		Name & Type & Cardinality & Definition
		\\\addlinespace
		\midrule\endhead
		dct:created & xsd:date & 0..1 & Date of creation of the resource.
		\\\addlinespace
		dct:modified & xsd:date & 0..1 & Date of modification of the resource.
		\\\addlinespace
		euvoc:endDate & xsd:date & 0..1 & End of the validity period. If a
		resource has an end date then it must be marked as deprecated.
		\\\addlinespace
		euvoc:startDate & xsd:date & 0..1 & Beginning of the validity period.
		\\\addlinespace
		owl:deprecated & xsd:boolean & 0..1 & States whether the resource is
		current or deprecated. By deprecating a resource, it means that it
		should not be used in new documents.
		
		Deprecation is a feature commonly used in versioning software to
		indicate that a particular feature is preserved for
		backward-compatibility purposes, but may be phased out in the future.
		\\\addlinespace
		rdf:value & rdfs:Literal & 1..1 & The literal form of the notation.
		\\\addlinespace
		dct:type & skos:Concept & 1..1 & Specify the context where a specified
		notation is considered unique.
		\\\addlinespace
		\bottomrule
		\addlinespace
		\caption{Properties of the euvoc:XlNotation class}
	\end{longtable}
}

\subsection{euvoc:XlNote}

Like the skosxl:Label class reifies SKOS label statements, XlNote
reifies SKOS note statements (i.e. skos:editorialNote, skos:example,
skos:historyNote, skos:definition, skos:scopeNote and skos:changeNote).
This class permits, if needed, to maintain the historical view of the
values and add additional provenance descriptions.

{
	%	\fontsize{10pt}{10pt}
	\footnotesize
	\selectfont%\tabcolsep=3pt % hold it local
	\begin{longtable}[c]{@{}p{3cm}p{2cm}p{2cm}p{7.8cm}@{}}
		\toprule\addlinespace
		Name & Type & Cardinality & Definition
		\\\addlinespace
		\midrule\endhead
		dct:created & xsd:date & 0..1 & Date of creation of the resource.
		\\\addlinespace
		dct:modified & xsd:date & 0..1 & Date of modification of the resource.
		\\\addlinespace
		dct:source & rdfs:Resource & 0..1 & A related resource from which the
		described resource is derived.
		
		The described resource may be derived from the related resource in whole
		or in part. Recommended best practice is to identify the related
		resource by means of a string conforming to a formal identification
		system.
		\\\addlinespace
		owl:deprecated & xsd:boolean & 0..1 & States whether the resource is
		current or deprecated. By deprecating a resource, it means that it
		should not be used in new documents.
		
		Deprecation is a feature commonly used in versioning software to
		indicate that a particular feature is preserved for
		backward-compatibility purposes, but may be phased out in the future.
		\\\addlinespace
		rdf:value & rdfs:Literal & 1..1 & The literal form of the note.
		\\\addlinespace
		\bottomrule
		\addlinespace
		\caption{Properties of the euvoc:XlNote class}
	\end{longtable}
}


\subsection{rdfs:Literal}

The class rdfs:Literal is the class of literal values such as strings
and integers. Property values such as textual strings are examples of
RDF literals.

\subsection{xsd:boolean}

The boolean data type is used to specify a true or false value.

\subsection{xsd:date}

The date data type is used to specify a date. The date is specified in
the following form ``YYYY-MM-DD'' where:

\begin{itemize}
\itemsep1pt\parskip0pt\parsep0pt
\item
  YYYY indicates the year
\item
  MM indicates the month
\item
  DD indicates the day
\end{itemize}

Note: All components are required!

\subsection{xsd:int}
