\section{Introduction}
\label{ariaid-title1}

This document defines the application profile for Legal Analisys
Methodology assets managed using
VocBench3\footnote{Stellato, A., Fiorelli,
	M., Turbati, A., Lorenzetti, T., Van Gemert, W.,Dechandon, D.,
	Laaboudi-Spoiden, C., Gerencser, A., Waniart, A.,Costetchi, E., and
	Keizer, J. (forthcoming). VocBench 3: a Collab-orative Semantic Web
	Editor for Ontologies, Thesauri and Lexicons.Semantic Web journal.
	\href{http://www.semantic-web-journal.net/content/vocbench-3-collaborative-semantic-web-editor-ontologies-thesauri-and-lexicons-1}{link}}. VocBench3 is
a web-based, multilingual, collaborative development platform for
managing OWL ontologies, SKOS(XL) thesauri and generic RDF datasets.
SKOS is a common data model for sharing and linking knowledge
organization systems via the Web. The SKOS data model provides a
standard, low-cost migration path for porting existing knowledge
organization systems to the Semantic Web. SKOS also provides a
lightweight, intuitive language for developing and sharing new knowledge
organization systems. It may be used on its own, or in combination with
formal knowledge representation languages such as the Web Ontology
language (OWL)\footnote{Bechhofer, S., \&
	Miles, A. (2009). SKOS Simple Knowledge Organization System Reference.
	\mbox{\url{https://www.w3.org/TR/skos-reference/}} }.

An Application Profile (AP) is a specification that re-uses terms from
one or more base standards, adding more specificity by identifying
mandatory, recommended and optional elements to be used for a particular
application, as well as recommendations for controlled vocabularies to
be used.

The Application Profile specified in this document is based on the
specification of the Simple Knowledge Organization System (SKOS). SKOS
is an \href{https://www.w3.org/TR/rdf11-concepts/}{RDF}\footnote{Wood, D., Lanthaler,
	M., \& Cyganiak, R. (2014). RDF 1.1 Concepts and Abstract Syntax.}
vocabulary designed to facilitate interoperability between controlled
vocabularies published on the Web as Linked Open Data. Additional
classes and properties from other well-known vocabularies are re-used
where necessary.

The work does not cover implementation issues like mechanisms to edit or
publish metadata assets and expected behaviour of systems implementing
the Application Profile other than what is defined in the Conformance
Statement.

The Application Profile is intended to facilitate controlled
vocabularies exchange and therefore the classes and properties defined
in this document are only relevant for the controlled vocabularies to be
exchanged; there are no requirements for communicating systems to
implement specific technical environments. The only requirement is that
the systems can export and import data in RDF in conformance with this
Application Profile.



