\section{Terminology used}
\label{ariaid-title12}

In the following sections, classes and properties are grouped under
headings "mandatory", "recommended" and "optional". These terms have the
following meaning.

\begin{itemize}
\item
  Mandatory class: a receiver of data MUST be able to process
  information about instances of the class; a sender of data MUST
  provide information about instances of the class.
\item
  Recommended class: a receiver of data MUST be able to process
  information about instances of the class; a sender of data MUST
  provide information about instances of the class, if it is available.
\item
  Optional class: a receiver MUST be able to process information about
  instances of the class; a sender MAY provide the information but is
  not obliged to do so.
\item
  Mandatory property: a receiver MUST be able to process the information
  for that property; a sender MUST provide the information for that
  property.
\item
  Recommended property: a receiver MUST be able to process the
  information for that property; a sender SHOULD provide the information
  for that property if it is available.
\item
  Optional property: a receiver MUST be able to process the information
  for that property; a sender MAY provide the information for that
  property but is not obliged to do so.
\end{itemize}

The meaning of the terms MUST, MUST NOT, SHOULD and MAY in this section
and in the following sections are as defined in RFC
2119\footnote{IETF. RFC 2119. Key
	words for use in RFCs to Indicate Requirement Levels.
	\url{http://www.ietf.org/rfc/rfc2119.txt}}.

In the given context, the term "processing" means that receivers must
accept incoming data and transparently provide these data to
applications and services. It does neither imply nor prescribe what
applications and services finally do with the data (parse, convert,
store, make searchable, display to users, etc.).

