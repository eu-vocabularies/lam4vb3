\section{Worksheet structure}

The excel file contains five worksheets: two of them define classes, two
of them define properties and one defines namespace prefix mappings. The
worksheet is composed of rows and columns. The rows roughly correspond
to descriptions of an identifiable entity/element, the columns
correspond to predicates (properties) in such descriptions while the
cell values to predicate objects. In every worksheet, the first row is
the header row. Each header cell denotes how the values in the
corresponding columns should be interpreted and processed (predicate
specification). Some denotations, as we will see below,
signify/represent (a) plain labels, (b) reference keys described in
another worksheet, (c) or encodings of functional links between
reference keys. The rest of the non empty rows represent distinct
definitions comprising value statements in all or most of the columns.

The columns that represent plain labels, provided lower case, serve a
descriptive purpose and most of the time the transformation script uses
them as such without much additional processing. Examples of such
columns can be found in ``LAM metadata worksheet'' which contains headers
such as ``Code''  ``controlled value property''  ``annotation1'' etc.

The columns that signify reference keys, provided in upper case, are
richer in meaning, which is provided in another worksheet. These sort of
columns are used in class definitions only, where the keys in the column
header function as references to property definitions. The
transformation script takes into consideration the definition linked to
the reference key and any additional relations and constraints when
processing the column values. Examples of such columns can be found in
 ``Classes Complete'' worksheet which contains headers such as ``RJ\_NEW'', ``CC'',  ``IF'',  ``EV'' etc.

The columns that encode functional links between reference keys (using
function notation), provided in upper case and round brackets, signify a
second order descriptions. They are used for encoding annotations of
values provided in another columns. For example the pair of columns ``EV'' and ``ANN\_COD(EV)'' means that the column ``EV'' contributes to the
description identified at the level of a row whereas the column
 ``ANN\_COD(EV)'' further extends the description provided by the column
 ``EV'' in the form of an annotation. The convention for such notations is
\emph{"KEY1(KEY2)"}, where \emph{KEY1} acts as a functor applied to
\emph{KEY2}; we read it ``KEY1 of KEY2'' or ``annotation of KEY2 with
KEY1''  The transformation script processes such column pairs in a
special manner tracking two levels of description identification, at the
level of the row and at the level of column value, taking into
consideration the definition linked to the reference keys, the link
between the reference keys and the implied constraints and
relationships.

The worksheet cells, which are slots formed at the intersection of a row
and a column provide, the values filling those slots. We distinguish few
kinds of cell values that are each controlled by a set of conventions. The value
types are as follows:

\begin{itemize}

\item
  Free text literal
\item
  Short URI notation
\item
  Controlled value
\end{itemize}

The free text literals are
Unicode\footnote{The Unicode Standard,
	Version 3, The Unicode Consortium, Addison-Wesley, 2000.
	\mbox{\url{http://www.unicode.org/unicode/standard/versions/}} } strings which
should be in Normal Form\footnote{Unicode Normalization
	Forms, Unicode Standard Annex \#15, Mark Davis, Martin Dürst.
	\href{http://www.unicode.org/unicode/reports/tr15/\%20}{http://www.unicode.org/unicode/reports/tr15/}}. The intended meaning
of short URI notation is specified by RFC 3986 on Uniform Resource
Identifiers\footnote{Berners-Lee, Fielding
	and Masinter (2005), RFC 3986 - Uniform Resource Identifier (URI):
	Generic Syntax. \url{https://tools.ietf.org/html/rfc3986}}. The
expected form is short reference URI \emph{"prefix:ID'' } where the
prefix (base URI) is formally defined in the document. The short URI
form is preferred to absolute (resolved) form URI, the latter being
discouraged form usage, nonetheless the transformation script is able to
identify and process them as accordingly. Both, the free text literals
and the short URI notations can be used as either (a) values of
properties (denoted by the column header) or (b) as property constraint
definitions. The interpretation depends of the column function described
below.

The last type of values, the controlled values, refer to are a
convention of specifying cardinality constraints in the class
definitions. This means that the cells with controlled values can not be
interpreted as property values but serve only as property constraints.
The conventions for cardinality constraints in LAM project are provided
in Table \ref{tab:cardinality}.

\begin{longtable}[!ht]{@{}p{3.4cm}p{3cm}p{2cm}p{4.5cm}@{}}
\toprule
Name & Cell value & Cardinality meaning & Alternative cell
values\tabularnewline
\midrule
\endfirsthead
\toprule
Name & Cell value & Cardinality meaning & Alternative cell
values\tabularnewline
\midrule
\endhead
mandatory & Y & 1..* & yes, y, according to text\tabularnewline
mandatory unique & YU & 1..1 &\tabularnewline
optional & O & 0..* &\tabularnewline
optional unique & OU & 0..1 &\tabularnewline
forbidden & N, \textless{}empty cell\textgreater{} & 0..0 & no,
n\tabularnewline
\bottomrule
\caption{Cardinality constraint conventions}
\label{tab:cardinality}
\vspace{-10pt}
\end{longtable}

The worksheet cells can contain commented values. It means that a cell
can contain a value (literal, URi or controlled) and in addition a
comment on that value. The value is separated from the comment by the
pipe (\textbar{}) character like this: ``\emph{value \textbar{}
comment}''  The transformation script uses the pipe character for
detecting commented values, and so this character should not be used for
any other purpose.

The worksheet cells can contain multiple values. The new value separator
is the new line character (CR/LF). This means that every new line of the
cell will be interpreted as a new value for the property indicated by
the column header.

\subsection{LAM class definition}
\label{sec:lam-class-definition}

The worksheet defining LAM classes plays central role in the LAM project
as it defines the document classes used in the legal analysis
methodology. It comprises of almost a hundred columns, which can be
grouped according to meaning and function they play in class
definitions. We distinguish the following functions:
\emph{identification}, \emph{description}, \emph{mappings} to other
classifications and \emph{property constraints}. All the columns are
headed with reference keys defined in the worksheet ``LAM metadata'' described below.

The URI column provides a universal identifier (as the title suggests)
for the row with values of the form \emph{"prefix:ID"}. The prefix is
defined in the prefix worksheet, described in Section \ref{sec:prefix-section}, and the ID part is automatically generated.

The \emph{description} columns, containing examples, keywords, comments
etc., represent human readable class descriptions. Their values
essentially are simple text literals. The \emph{mapping} columns provide
correspondences between LAM classes and other classifications, in this
case the CDM ontology, the Resource Type authority table, and CELEX
classification. These mappings to other classifications are intended for
manually or eventually automatically determining and/or validating the LAM
class to which a legal document belongs.

The rest of the columns represent \emph{property constraints}. In the
context of class definition, property constraints mean that instances of
the defined class must respect the specified constraint. The constraints
are provided either as a literal value, URI or cardinality specification
(see Table \ref{tab:cardinality}). In case of literal or URI values, the constraints mean that the
instances of the class being defined must provide property statements
with exactly same values. If there are multiple values, then the default
interpretation is that of alternative values either of which should be
found among those provided in the instance data. In case of cardinality
specifications, the interpretation is on the number of times a property
is employed for a given instance. For example, mandatory properties must
be employed once or multiple times, having the minimum cardinality set
to one, while optional unique properties may be employed at most once
with minimum cardinality set to zero and maximum to one. The cardinality
constraints do not provide any indications about the range of values
used of a given property.

Some constraints headed by a function notation represent annotation
constraints on a property. For example the column ``EV'' (date of end of
validity) is annotated with ``ANN\_COD'' (annotation: comment on date)
column written as ``ANN\_COD(EV)''  The values in this column represent
cardinality constraints on the comment on date property. For example if
there is a ``O'' value provided in ``ANN\_COD(EV)'' column then, whenever
there is an end of date property employed on an instance then, that
value, may optionally be annotated with a comment on date.

The last three columns ``Classification level 1'' to ``Classification level
3'' provide a classification structure for the defined documents as
originally specified in the LAM documentation.

\subsection{LAM property definition}

The LAM property definition worksheet defines the meaning to the columns
used in the class definition worksheet(s). As mentioned in the
introduction above, the columns roughly correspond to
predicates/properties in the LAM model and are locally identified by a
unique ``Code'' (usually in capital letters). The same codes are used as a
reference values in the column headers of the class definition worksheet
indicating which property shall be used from the model for each column.
The ``Code'' is used to generate the LAM property URI used in the formal
statements.

The property definition worksheet is structured as follows. The ``Label'' column provides a human friendly property title; the ``Definition'' provides a human readable property meaning. ``Analytical methodology'' is
a description of how the property contributes to the LAM practice.
 ``Specific cases'' and ``Comments'' provide examples, exceptions and
additional comments related to property usage. Example values for these
columns are provided in Table \ref{tab:ex1}.

{
\fontsize{10pt}{10pt}
%\footnotesize
\selectfont%\tabcolsep=3pt % hold it local
\begin{longtable}[c]{@{}p{2.73cm}p{1.7cm}p{2.5cm}p{1.62cm}p{4.75cm}@{}}
	\toprule\addlinespace
	URI  & Label & Property & Controlled value & Definition
	\\\addlinespace
	\midrule\endhead
	lamd:EXAMPLE & English example & skos:example@en & &
	English Example. This field used in the cataloguing methodology for
	information purposes.
	\\\addlinespace
	lamd:CDM\_CLASS & CDM class & lam:cdm\_class & & Class
	or subclass according to CDM.
	\\\addlinespace
	lamd:FM & Type of act & cdm:resource-type & at:resource-type &
	Type of act is usually mentioned in the title.
	\\\addlinespace
	\bottomrule
	\addlinespace	
	\caption{Example of human readable fields in LAM property definition}
	\label{tab:ex1}
	\vspace{-10pt}
\end{longtable}
}



%\begin{table}[]
%	\resizebox{\textwidth}{!}{%
%	\begin{tabularx}{1.8\textwidth}{>{\small}X>{\hsize=.5\hsize}XX>{\hsize=.5\hsize}X>{\hsize=2\hsize}X}
%		\toprule
%		URI                 & Label       & property          & controlled value & Definition                                                                                \\ \midrule
%		lamd:md\_EXAMPLE\_EN & EN example  & skos:example@en   &                  & English Example. This field used in the cataloguing methodology for information purposes. \\
%		lamd:md\_CDM\_CLASS  & CDM class   & lam:cdm\_class    &                  & Class or subclass according to CDM.                                                       \\
%		lamd:md\_FM          & Type of act & cdm:resource-type & at:resource-type & Type of act is usually mentioned in the title.                                            \\ \bottomrule
%	\end{tabularx}
%	}
%\end{table}


The ``property'' column specifies URI of the equivalent property formally
defined in CDM ontology (other namespaces are also accepted). If there
is a range constraint to, for example, a controlled vocabulary then it
is indicated in the ``controlled value property''.

The ``property type'' column indicate a formal constraint on how the property can be instantiated. Two options are available: ``object property'', which means that the range is always an URI and ``data property'', which means that the range is always a literal. This specifications corresponds to OWL2 semantics of owl:ObjectProperty and owl:DataProperty. If left unspecified, the fallback is the rdf:Property semantics, but this option is strongly discouraged. 

As mentioned, in Section \ref{sec:lam-class-definition}, some CDM properties are annotated to provide
extra information. The columns ``annotation\_1'' to ``annotation\_7'' specify which CDM annotation properties may be used for the defined
property. Columns ``controlled value\_annotation\_1'' to ``controlled
value\_annotation\_7'' provide range constraints on the corresponding
property. The values in columns ``annotation\_11'' to ``annotation\_71'' are
automatically generated and do not provide any additional information,
but play a technical role for the transformation script, providing a
mapping between the URI of the CDM property and the URI of the LAM
property. The translation pairs are provided in an auxiliary ``mappings'' worksheet.

The last two columns ``Classification level 1'' and ``Classification level
2'' provide a classification structure for the defined properties as
originally specified in the LAM documentation.



\subsection{CELEX class and property definition}

This worksheet aims at capturing the description of CELEX classes
following the logic that has been used to allocate CELEX numbers since
the setting-up of the EUR-Lex database (formerly known as CELEX). The
CELEX classes are defined as a combination of DTS, DTT, DTA and OJ\_ID
columns (described below) and are structured on three levels:

\begin{enumerate}
\item
  DTS classes (CELEX sectors)
\item
  DTS*DTT (power product) classes. These classes corresponds to rows in
  the sector tables describing DTTs of the sectors.
\item
  DTS*DTT*OJ\_ID (power product) classes. These classes correspond to
  cells in the sector tables describing DTTs for each of the three
  OJ\_IDs of the sector.
\end{enumerate}

\begin{longtable}[!ht]{@{}p{3.2cm}p{2cm}p{2cm}p{2cm}p{2cm}@{}}
	\toprule
	DN & DTS & DTA & DTT & DTN\tabularnewline
	\midrule
	\endfirsthead
	\toprule
	DN & DTS & DTA & DTT & DTN\tabularnewline
	\midrule
	\endhead
	32019R0001 & 3 & 2019 & R & 0001\tabularnewline
	C2019/123 & C & 2019 & \textless{}empty\textgreater{} &
	123\tabularnewline
	52014AE1723 & 5 & 2014 & AE & 1723\tabularnewline
	\bottomrule
	\caption{Examples of CELEX number composition}
	\label{tab:ex2}
	\vspace{-10pt}
\end{longtable}

\begin{equation}
	\emph{DN =
		\textless{}DTS\textgreater{} \textless{}DTA\textgreater{}
		\textless{}DTT\textgreater{} \textless{}DTN\textgreater{}}
	\label{eq:eq1}
\end{equation}

The CELEX number anatomy is provided in Formula \ref{eq:eq1}. Examples of
CELEX numbers and how they are composed can be seen in Table \ref{tab:ex2}. Where the
column name acronyms mean the following:

\begin{itemize}

\item
  DN - the specific instance of CELEX number. Legal document metadata.
\item
  DTS - Sector
\item
  DTT - Document type
\item
  DTA - The year
\item
  DTN - The number
\end{itemize}



Properties describing class at each level are as follows.

\textbf{Level I classes:}

\begin{itemize}

\item
  label
\item
  code (=DTS)
\item
  DTS
\item
  definition
\item
  scope note (optional)
\item
  comments (optional)
\end{itemize}

\textbf{Level II classes} (same as above plus additionally):

\begin{itemize}

\item
  *code (=DTS*DTT)
\item
  DTT
\item
  author
\end{itemize}

\textbf{Level III classes} (same as above plus additionally):

\begin{itemize}

\item
  *code (=DTS*DTT*OJ\_ID)
\item
  OJ\_ID (either \emph{"OJC"}, \emph{"OJL"} or \emph{"EuroLex}")
\item
  DTA source of .. (as indicated in CELEXspecification documentation
  section 1 on general rules)
\item
  DTN source of .. (as indicated in CELEXspecification documentation
  section 1 on general rules)
\end{itemize}

The CELEX property definition worksheet, just like the one for LAM
properties, defines a set of properties used in CELEX class definition.
They are primarily CELEX composition properties but in the current project a few auxiliary properties are used.

\subsection{Namespace prefix definitions}
\label{sec:prefix-section}

This worksheet provides a mapping between the LAM property definitions
and CDM ontology (or another namespace). This worksheet is auxiliary and
has a technical role aiding the transformation script. The worksheet is
composed of two columns, first the URI of the CDM ontology and the
second one the URI of the LAM property.

